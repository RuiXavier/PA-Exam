\documentclass[11pt, a4paper]{article}
\usepackage[utf8]{inputenc}
\usepackage{geometry}
\usepackage{enumitem}
\usepackage{amsmath}
\usepackage{amssymb}
\usepackage{titlesec}

% Geometry settings for readable notes
\geometry{left=2.5cm, right=2.5cm, top=2.5cm, bottom=2.5cm}

% Formatting titles to save space
\titleformat{\section}{\large\bfseries}{\thesection}{1em}{}
\titlespacing*{\section}{0pt}{10pt}{5pt}

\title{\textbf{Oral Exam Notes: A Compositional Deadlock Detector for Android Java}\\
\large Based on Brotherston et al. (ASE 2021)}
\date{}
\author{}

\begin{document}

\maketitle

\section*{1. Motivation \& Context (Approx. 1 Minute)}
\textit{Goal: Establish why this research exists and the specific industrial constraints.}

\begin{itemize}
    \item \textbf{The Problem:} Concurrency is ubiquitous in Android (UI thread + background threads), making deadlocks a core reliability issue that freezes apps.
    \item \textbf{The Industrial Setting:} The authors target Facebook's massive Android codebases (tens of millions of LoC) with thousands of daily commits.
    \item \textbf{The Challenge:} Standard whole-program analysis is too slow for Continuous Integration (CI). They need a tool that runs at \textbf{code-review time} (minutes, not hours) and focuses on \textit{actionable} reports over theoretical perfection.
    \item \textbf{Research Gap:} Existing tools either require the whole program or prioritize soundness (proving absence of deadlocks) at the cost of high false-positive rates, which developers ignore.
\end{itemize}

\section*{2. Theoretical Framework: Critical Pairs (Approx. 1.5 Minutes)}
\textit{Goal: Explain the "Core Contribution" — the mathematical model.}

\begin{itemize}
    \item \textbf{Abstract Language:} The paper models Android Java as an abstract language with \textit{balanced re-entrant locks} (like \texttt{synchronized} blocks) and non-deterministic control.
    \item \textbf{Critical Pairs:} The central innovation is the concept of a \textbf{Critical Pair}. 
    \item \textbf{Definition:} For a thread $T$, a critical pair is a tuple $(X, l)$, meaning: "There exists an execution where the thread attempts to acquire lock $l$ while \textit{already holding} the set of locks $X$".
    \item \textbf{The Deadlock Theorem:} The authors prove that a deadlock occurs if and only if two threads have conflicting critical pairs:
    \[
    (X_1, l_1) \in \text{Crit}(C_1) \quad \text{and} \quad (X_2, l_2) \in \text{Crit}(C_2)
    \]
    Such that each thread holds the lock the other needs ($l_1 \in X_2$ and $l_2 \in X_1$) and their held locks do not overlap ($X_1 \cap X_2 = \emptyset$).
    \item \textbf{Complexity:} This problem is proven to be decidable and in \textbf{NP}.
\end{itemize}

\section*{3. Implementation: Compositionality (Approx. 1.5 Minutes)}
\textit{Goal: Explain how the theory becomes a tool that scales.}

\begin{itemize}
    \item \textbf{Abstract Interpretation:} The tool calculates critical pairs using abstract interpretation. It computes a summary for each method.
    \item \textbf{Compositionality is Key:} Because the analysis is compositional, they do not need to re-analyze the whole app. When a developer submits a code change, the tool only analyzes the changed methods and their dependencies.
    \item \textbf{Handling Android Specifics:} 
    \begin{itemize}
        \item They map Java \texttt{synchronized} blocks to the balanced locks in the abstract language.
        \item They use heuristics for thread identity (e.g., \texttt{@UiThread}, \texttt{@WorkerThread}) to reduce false positives.
    \end{itemize}
    \item \textbf{Tooling:} The implementation is called \texttt{Starvation}, integrated into the \textbf{INFER} static analysis framework.
\end{itemize}

\section*{4. Results \& Evaluation (Approx. 0.5 Minutes)}
\textit{Goal: Prove it actually works in the real world.}

\begin{itemize}
    \item \textbf{Deployment:} Deployed on all Android commits at Facebook for over 2 years.
    \item \textbf{Performance:} Fast analysis times—median of 90 seconds per commit.
    \item \textbf{Impact:} Detected over 500 deadlock reports. Crucially, developers fixed $\approx 54\%$ of these reports, indicating high trust and "actionability".
\end{itemize}

\section*{5. Conclusion \& Critique (Approx. 0.5 Minutes)}
\textit{Goal: Summarize and show critical thinking.}

\begin{itemize}
    \item \textbf{Pros:} Successfully bridges the gap between theoretical guarantees (decidability) and industrial scale. The use of summaries makes it maintainable.
    \item \textbf{Limitations/Cons:}
    \begin{itemize}
        \item It relies heavily on "balanced locking" (structured locking); unstructured locking breaks the model.
        \item Compositionality can lead to false negatives if a deadlock spans across parts of the call graph not currently being analyzed.
    \end{itemize}
\end{itemize}

\end{document}